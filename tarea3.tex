\documentclass[conference]{IEEEtran}

\usepackage{graphicx}
\usepackage{subcaption}
\usepackage{amsmath} % assumes amsmath package installed
\usepackage{subcaption}
\usepackage{amssymb}
\usepackage{array}
\usepackage{tikz,circuitikz}
\usetikzlibrary{fit}
\usetikzlibrary{arrows.meta,positioning}
\usepackage{blindtext}
\usepackage{color}
\usepackage{siunitx}
\usepackage{float} % enables [H] float placement
\usepackage{booktabs} % for \toprule, \midrule, \bottomrule
\usepackage[utf8]{inputenc}

\renewcommand{\tablename}{Tabla}

\def\BibTeX{{\rm B\kern-.05em{\sc i\kern-.025em b}\kern-.08em
    T\kern-.1667em\lower.7ex\hbox{E}\kern-.125emX}}


\usepackage{mwe}
\usepackage{fancyhdr}
\fancypagestyle{firststyle}
{
	\fancyhf[C]{\fontsize{8}{10} \selectfont \textit{} }
	\fancyfoot[C]{}
}


\hyphenation{op-tical net-works semi-conduc-tor}


\begin{document}
%
% paper title
% Titles are only capitalized in the first letter.
% Linebreaks \\ can be used within to get better formatting as desired.
% Do not put math or special symbols in the title.
\title{Document title}

\author{\IEEEauthorblockN{1\textsuperscript{st} Given Name Surname}
\IEEEauthorblockA{\textit{dept. name of organization (of Aff.)} \\
\textit{name of organization (of Aff.)}\\
City, Country \\
email address}
\and
\IEEEauthorblockN{2\textsuperscript{nd} Given Name Surname}
\IEEEauthorblockA{\textit{dept. name of organization (of Aff.)} \\
\textit{name of organization (of Aff.)}\\
City, Country \\
email address}
\and
\IEEEauthorblockN{3\textsuperscript{rd} Given Name Surname}
\IEEEauthorblockA{\textit{dept. name of organization (of Aff.)} \\
\textit{name of organization (of Aff.)}\\
City, Country \\
email address}
}


\maketitle

\thispagestyle{firststyle}
\renewcommand{\headrulewidth}{0in}
\pagestyle{empty}


\pagestyle{fancy}
\chead{\fontsize{8}{10} \selectfont \textit{} }
\pagenumbering{gobble}



% As a general rule, do not put math, special symbols or citations
% in the abstract
\begin{abstract}
This document is a model and instructions for \LaTeX.
This and the IEEEtran.cls file define the components of your document [title, text, heads, etc.]. *CRITICAL: Do Not Use Symbols, Special Characters, Footnotes, 
or Math in Document title or Abstract.
\end{abstract}


\IEEEpeerreviewmaketitle



\section{Introduction}
This document is a model and instructions for \LaTeX.
Please observe the report page limits. 

\begin{figure}[ht]
    \centering
    \begin{circuitikz}[american]
        \ctikzset{
          diodes/scale=0.4, sources/scale=0.5, resistors/scale=0.5, inductors/scale=0.75, capacitors/scale=0.75}

        \draw (-0.75,2.65) to[sV,-*, l={\scriptsize $v_a$},label distance=-2pt,i_>={\scriptsize $i_a$}] (1,2.65)
            node[pos=0.5, right, yshift=  80pt, xshift = 6pt]{\scriptsize $+$}
            node[pos=0.5, left, yshift=80pt, xshift = 1pt]{\scriptsize $-$};
        \draw (-0.75,2.00) to[sV, l={\scriptsize $v_b$},label distance=-2pt,i_>={\scriptsize $i_b$}] (1,2.00)
            node[pos=0.5, right, yshift=  61pt, xshift = 6pt]{\scriptsize $+$}
            node[pos=0.5, left, yshift=61pt, xshift = 1pt]{\scriptsize $-$};
        \draw (-0.75,1.35) to[sV, l={\scriptsize $v_c$},label distance=-2pt,i_>={\scriptsize $i_c$}] (1,1.35)
            node[pos=0.5, right, yshift=42pt, xshift = 6pt]{\scriptsize $+$}
            node[pos=0.5, left, yshift=42pt, xshift = 1pt]{\scriptsize $-$};

        \draw (1,0)   to[Ty*, l={\scriptsize $T_4$}, label distance=-10mm] (1,1.35)
                  to[short] (1,2.65)
                  to[Ty*, l={\scriptsize $T_1$}, label distance=-10mm] (1,4);

        \draw (1.75,0) to[Ty*, l={\scriptsize $T_6$}, label distance=-10mm] (1.75,1.35)
                  to[short] (1.75,2.65)
                  to[Ty*, l={\scriptsize $T_3$}, label distance=-10mm] (1.75,4);

        \draw (2.5,0)   to[Ty*, l={\scriptsize $T_2$}, label distance=-10mm] (2.5,1.35)
                  to[short] (2.5,2.65)
                  to[Ty*, l={\scriptsize $T_5$}, label distance=-10mm] (2.5,4);

        \draw (4,0) to[R, l={\scriptsize $R$}, v_<={\scriptsize $v_R$}, f<^={\scriptsize $i_R$}] (4,4);
        \draw (1,4) -- (2,4) to[short, i>^={\scriptsize $i_{oi}$}] (4,4);
        \draw (1,0) to[short] (4,0);
        \draw (-0.75,1.35) -- (-0.75,2.65);
        \draw (-0.75,2) node[left]{\scriptsize $n$};
        \draw (1,1.35) to[short,-*] (2.5,1.35);
        \draw (1,2) to[short,-*] (1.75,2);
        \draw (2.5,0) -- (2.5,4)
            node[pos=0.5, right]{\scriptsize $v_{oi}$}
            node[pos=0.9, right]{\scriptsize $+$}
            node[pos=0.1, right]{\scriptsize $-$};

    \end{circuitikz}
\end{figure}

Lo voltajes de línea a nueutro se definen como:
\begin{equation*}
	\begin{aligned}
		v_a(t) &= V_{max}\sin(\omega t) \\
		v_b(t) &= V_{max}\sin(\omega t - 120^\circ) \\
		v_c(t) &= V_{max}\sin(\omega t + 120^\circ)
	\end{aligned}
\end{equation*}

\section{Estados asumidos}

	En un instante dado las tres tensiones de fase $v_a,v_b,v_c$ se ordenan por su valor instantáneo; con SCR ideales y sin solape, siempre conducen dos dispositivos: el superior de la fase con mayor potencial y el inferior de la fase con menor potencial, de modo que el voltaje de salida es siempre un línea–a–línea $v_{oi}(t)=v_{\max}(t)-v_{\min}(t)$. Se considera además el estado sin conducción $S_0$ cuando aún no se ha aplicado el pulso de compuerta (ángulo $\alpha$) o la red no supera el umbral de disparo. Así, los siete estados son: 
	
	\begin{itemize}
		\item $S_0$: ningún SCR conduce, $v_{oi}=0$.
		\item $S_1$ si $v_a\ge v_b\ge v_c$, $T_1$ y $T_6$ (ON), $v_{oi}=v_a-v_b\equiv v_{ab}$.
		\item $S_2$ si $v_a\ge v_c\ge v_b$, $T_1$ y $T_2$ (ON), $v_{oi}=v_a-v_c\equiv v_{ac}$.
		\item $S_3$ si $v_b\ge v_a\ge v_c$, $T_3$ y $T_4$ (ON), $v_{oi}=v_b-v_a\equiv v_{ba}$.
		\item $S_4$ si $v_b\ge v_c\ge v_a$, $T_3$ y $T_2$ (ON), $v_{oi}=v_b-v_c\equiv v_{bc}$.
		\item $S_5$ si $v_c\ge v_a\ge v_b$, $T_5$ y $T_4$ (ON), $v_{oi}=v_c-v_a\equiv v_{ca}$.
		\item $S_6$ si $v_c\ge v_b\ge v_a$, $T_5$ y $T_6$ (ON), $v_{oi}=v_c-v_b\equiv v_{cb}$. 
	\end{itemize}

	Las transiciones entre estados ocurren cuando dos tensiones de fase se igualan (cruces $v_a=v_b$, $v_b=v_c$ o $v_c=v_a$), lo que en el caso senoidal balanceado sucede cada $60^\circ$ eléctricos; el inicio efectivo de cada estado conductor queda retrasado por el ángulo de disparo $\alpha$. Dado que la carga es estrictamente resistiva, no hay almacenamiento de energía ni memoria de estado: la corriente sigue instantáneamente a la tensión en cada tramo y vale simplemente
		\begin{equation*}
			i(t)=\frac{v_{oi}(t)}{R},
		\end{equation*}
	anulándose exactamente en los instantes de cambio de par, por lo que el apagado de los SCR es natural en cada frontera de \(60^\circ\).

% (p. ej., \(T_1\) a \(\omega t=\tfrac{\pi}{6}+\alpha\), \(T_2\) a \(\tfrac{\pi}{2}+\alpha\), etc.)

\begin{table}[h!]
	\centering
	\small
	\begin{tabular}{@{}cc@{}}
		\toprule
		Estado & Intervalo (grados) \\ \midrule
	    $S_{1}$ & $[30^\circ+\alpha,\;90^\circ+\alpha)$ \\
		$S_{2}$ & $[90^\circ+\alpha,\;150^\circ+\alpha)$ \\
		$S_{3}$ & $[150^\circ+\alpha,\;210^\circ+\alpha)$ \\
		$S_{4}$ & $[210^\circ+\alpha,\;270^\circ+\alpha)$ \\
		$S_{5}$ & $[270^\circ+\alpha,\;330^\circ+\alpha)$ \\
		$S_{6}$ & $[330^\circ+\alpha,\;390^\circ+\alpha)$ \\ \bottomrule
	\end{tabular}
	\caption{Intervalos de los estados asumidos del rectificador trifásico semicontrolado (ángulos en grados; $\alpha$ en grados).}
	\label{tab:estados_grados}
\end{table}

\noindent En cada intervalo de $60^\circ$ la secuencia de fases (desfasadas $120^\circ$) produce una fase \emph{más positiva} y otra \emph{más negativa}; el par de tiristores del sector conecta dicha fase máxima con la mínima, por lo que $v_{oi}$ en ese tramo es la diferencia línea-a-línea entre ellas. El ángulo de disparo $\alpha$ (expresado en grados) no altera el orden relativo de las fases: simplemente \emph{desplaza} el inicio de cada sector (p. ej. $S_{1}$ comienza en $30^\circ+\alpha$, $S_{2}$ en $90^\circ+\alpha$, etc.). En un modelo ideal por tramos cada sector dura $60^\circ$; con inductancias reales pueden aparecer solapamientos momentáneos durante los cruces.

\section{Conclusions}
The conclusion goes here.

\begin{thebibliography}{plain}
\bibitem{b1} G. Eason, B. Noble, and I. N. Sneddon, ``On certain integrals of Lipschitz-Hankel type involving products of Bessel functions,'' Phil. Trans. Roy. Soc. London, vol. A247, pp. 529--551, April 1955.
\bibitem{b2} J. Clerk Maxwell, A Treatise on Electricity and Magnetism, 3rd ed., vol. 2. Oxford: Clarendon, 1892, pp.68--73.
\bibitem{b3} I. S. Jacobs and C. P. Bean, ``Fine particles, thin films and exchange anisotropy,'' in Magnetism, vol. III, G. T. Rado and H. Suhl, Eds. New York: Academic, 1963, pp. 271--350.
\bibitem{b4} K. Elissa, ``Title of paper if known,'' unpublished.
\bibitem{b5} R. Nicole, ``Title of paper with only first word capitalized,'' J. Name Stand. Abbrev., in press.
\bibitem{b6} Y. Yorozu, M. Hirano, K. Oka, and Y. Tagawa, ``Electron spectroscopy studies on magneto-optical media and plastic substrate interf	ace,'' IEEE Transl. J. Magn. Japan, vol. 2, pp. 740--741, August 1987 [Digests 9th Annual Conf. Magnetics Japan, p. 301, 1982].
\bibitem{b7} M. Young, The Technical Writer's Handbook. Mill Valley, CA: University Science, 1989.
\end{thebibliography}


\end{document}